\chapter{Sentiment Analysis across Different Cities}
In this chapter we propose and explain the sentiment analysis across different cities.
\section{Overview}
Figure \ref{fig:overview} describes the overview of the system.
The system takes collected tweets by city as input.
They are searched by the Twitter API, specifically with the place ID, which explained in Section \ref{sec:placeid}.
Tweets are searched by each language by specifying language parameter of the Twitter search API.
Before the tweets are labelled by the classifiers, they are pre-processed.
One of the reasons of the preprocessing is to replace an emoji unicode in them with the explanation of the emoji.
%This replacement is applied to all of the languages.
The classifiers are built by machine learning and they require training data of annotated tweets, which are labelled positive, negative or neutral.
In this paper we use three kinds of classifiers.
Two of them are Support Vector Machine (SVM) for multi class, one versus one and one versus all and the other one is Random Forest.
During classification, the parameters of each classifier are optimised by cross validation.
In the end, classifiers label input tweets as one of sentimental values, which are positive, negative or neutral.

The novelty of this system is taking into account location information.
The location information make an analysis in each location possible.
And we perform the sentiment analysis in multi-language and this unveils more aspects than monolingual analysis.

We explain the details of each process in the following sections.
\begin{figure}
	\centering
	\includegraphics[width=10cm]{./fig/overview2.png}
	\caption{Overview of sentiment analysis by city}
	\label{fig:overview}
\end{figure}


\section{Preprocessing}
Not only on Twitter but also on SNS as a whole many people are using emoji nowadays \cite{emoji}.
Besides emojis are used to express not just face expressions.
There are other categories that are not face mark, such as Animals \& Nature, Food \& Drink, Flags etc.
Sometimes it takes an important role of a sentence.
For example, Figure \ref{fig:crying} shows a tweet with an emoji of smiley with tears and the same tweet without the emoji.
The one with the emoji looks to express something positive but the other one looks to have negative sentiment.
Like this comparison, just one emoji can influence a sentiment of a sentence, tweet.
Thus, emojis should not be discarded in sentiment analysis.
\begin{figure}
	\centering
	\scalebox{0.4}[0.4]{
	\includegraphics[width=10cm]{./fig/crying.png}
	}
	\caption{Comparison of a tweet with or without an emoji}
	\label{fig:crying}
\end{figure}


To detect all of the unicodes of emoji, we use a python module available on github \cite{emoji_package}. 
This module is originally for output emojis on terminal.
Figure \ref{fig:thumbs_up} shows an example of how to print an emoji.
The module also recognises each emoji unicode using a dictionary in the module.
A part of the dictionary is depicted in Figure \ref{fig:emoji_dict}.
Therefore we implement a script to detect unicodes of emoji and to replace them with the explanations of each emoji.


\begin{figure}
	\centering
	\scalebox{1.0}[1.0]{
	\includegraphics[width=10cm]{./fig/emoji_package.png}
	}
	\caption{Example of how to print an emoji on terminal}
	\label{fig:thumbs_up}
\end{figure}


\begin{figure}
	\centering
	\scalebox{1.0}[1.0]{
	\includegraphics[width=10cm]{./fig/emoji_dictionary.png}
	}
	\caption{A part of a dictionary in emoji module}
	\label{fig:emoji_dict}
\end{figure}


We replace emoji unicodes by the script using the emoji module, specifically the dictionary in the module.
%The script detects an unicode of an emoji in a tweet and 
%In Figure \ref{fig:emoji_replacement}, an emoji at the end of the sentence is replaced with the explanation of the emoji.
The script detects an unicode of an emoji and replaces it with the description as Figure \ref{fig:emoji_replacement} shows.
In this way, the tweet can have more meaningful characters for classifiers than the unicode, which is ``u\textbackslash U0001F625"
In another way of saying, the sentence in Figure \ref{fig:emoji_replacement} gets more information because of the replacement.

However, English is only available for replacement in this module itself.
Since we perform a multi-language sentiment analysis, we need to handle the other languages.
In a website \cite{emoji_explanation} explanations of emoji are listed in each category in 12 languages. 
Not only European languages, even Asian languages, such as Japanese, Chinese, Arabic etc., are also available.
Based on the descriptions, we manually make CSV file like below.
Each line in the CSV file contains a key of an emoji and the descriptions of the emoji in English, French, German and Spanish.
We replace unicodes with the descriptions in each language.
Tweets are processed in each language.
So, the replacement in all of the four languages becomes possible by picking one of the descriptions depending on the language. 
%This replacement is available in all of the four languages.
\begin{description}
%{A part of emoji description CSV file}
%English, Francais, Deutsch, Espa\~{n}ol, Portugu\^{e}s
%keys,meaning,sens,Bedeutung,significado,significado
	\item[key] English, French, German, Spanish
	\item[:soccer\_ball:] Soccer Ball, Ballon de foot, Fu{\ss}ball, Bal\'on de f\'utbol
	\item[:basketball\_and\_hoop:] Basketball And Hoop, Ballon de basket et panier, Basketball und Korb,Bal\'on de baloncesto y canasta
	\item[:american\_football:] American Football, Football am\'ericain, American Football, F\'utbol americano

	\item[:baseball:]Baseball, Balle de base-ball, Baseball, Pelota de b\'eisbol
\end{description}

When a user mentions something from an URL or other users, tweet contains URLs and user names as well.
On Twitter, mentioning other users makes communication between users possible.
And users puts an URL to share the website with other users.
%It is also possible that a user is mentioned in tweets of other users. 
But they are not really necessary for a sentiment analysis because we do not take into account of user information and shared URL in tweets.
Therefore we replace URL and user names in tweets with strings, ``~http" and ``@user" respectively.
The reason to use these strings is because originally they are used in the training dataset described in next section.

\begin{figure}
	\centering
	\includegraphics[width=10cm]{./fig/emoji_replacement2.png}
	\caption{How an emoji is replaced in a tweet}
	\label{fig:emoji_replacement}
\end{figure}


\section{Dataset}\label{sec:dataset}
what dataset is used in this research ?
We do not prepare training dataset ourselves but instead take it from other works.
The number of dataset of each language are listed in Table \ref{tab:dataset}.
Spanish dataset is much larger than the others because this is the only dataset from \cite{dataset_spanish} and the others are from \cite{dataset}.
All of the datasets are annotated by people and importantly the tweets in the datasets are labelled sentimental values.
Therefore they can be considered to be trustful and we believe the amount of the datasets are large enough.

\begin{table}[ht]
	\caption{Number of dataset of each language}
	\centering
	\begin{tabular}{|c|r|} \hline
	Language & \# of dataset \\ \hline \hline
	English & 7,200  \\ \hline
	French & 1,797  \\ \hline
	German & 1,800  \\ \hline
	Spanish & 68,000  \\ \hline
	\end{tabular}
	\label{tab:dataset}
\end{table}


\section{Implementation of Classifiers}
Classifiers.
implement by scikit-learn API in Python 2.7.12

We employ three kinds of classifiers in this paper.
Originally SVM is a binary classification method but it can be used for multi-class classification as well.
There are two ways for multi-class classification SVM, one versus one and one versus all.
One versus one compares two classes of N classes and constructs a hyperplane while one versus all compares one of N classes and the others.
Random forest is an ensemble learning method that samples random data and constructs decision trees.
It classifies based on the results of the multiple decision trees. 
We apply the API of scikit-learn \cite{scikit} to build these classifiers in Python 2.7.12.


