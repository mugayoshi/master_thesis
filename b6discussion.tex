\chapter{Discussion}
In this chapter we discuss about the results shown in Section \ref{sec:result}.
\section{Classification Result}
We discuss about the classification results by each language in this section and summarise for all of the languages in the end.
\subsection{English}
Apparently in all of the cases more tweets are labelled as neutral than the others.
%Besides tweets that are classified as neutral are more than half of the whole.
The minimum percentage of neutral tweets, 49.59 \%, is the result of San Francisco by One vs. One SVM classifier (Table \ref{tab:result_sf_en}).
Except this case, more than half of the all of the tweets consist of tweets classified as neutral in every case.
And the maximum of neutral label, 89.30 \%, is the result of Buenos Aires by Random Forest classifier (Table \ref{tab:result_buenosaires_en}).
The minimum percent of both positive and negative is also this casei, 6.85 \%.
The maximum percentage of positive is London One vs. One SVM, 23.66 \% (Table \ref{tab:result_london_en}) and the maximum of negative is San Francisco One vs. One SVM, 26.92 \%.


For most of the cities there are more positive tweets than negative.
The exceptions are London One vs. All SVM, all case of New York, Quebec One vs. All SVM and all case of San Francisco.
Interestingly both of the two cities that have more negative tweets than positive in all case are in the same country, the U.S.
Also, San Francisco One vs. One SVM (26.92 \%) and New York One vs. One SVM (26.90 \%) are the two highest percentage of negative label.
According to these results, it seemed more users tweeted negative sentences both in the two cities than other cities.
%We collected tweets after presidential election in November 2016 and .
\subsection{French}
more negative tweets ?

difference between cities
\subsection{German}
there are more neutral tweets than other languages.

difference between cities
\subsection{Spanish}
almost no neutral tweets.

difference between cities

\subsection{Overall}\label{sec:discussion_overall}
We can point out clearly Random Forest classifier tends to choose neutral more than SVM classifiers except Spanish.
This is applied to all of the results.
One of the reason for this is that there are more neutral tweets than others in the training dataset.
As Table \ref{tab:dataset2} shows, the English data labelled as positive, negative and neutral with agreement by more than two people are 1,595, 998 and 4,238 respectively for instance.

We added extra data to the training dataset to prepare other training dataset for English, French and German.
Similar to \cite{dataset}, we added data labelled based on emoticons (positive and negative only).
For example we labelled a tweet that contains one of smile emoticons as positive and vice versa for negative emoticons, such as :(, :/ and so on.
As for Spanish, we eliminated positive and negative data from the Spanish dataset to adjust the size of each label. 

It turned out that the results of the modified training dataset were different from the original.
We found out that One vs. One SVM classifier trained on English data with more positive tweets labels tweets as positive more often than the original training dataset. 
Also it figured out that even Random Forest classifier trained on the modified Spanish data with less positive and negative tweets choose neutral much more often.
Thus, the classification result depends on training dataset, especially how many data of each label exist.

There are two reasons that we decided not to apply the modified training dataset.
At first it was not sure how common to use emoticons on Twitter.
So, we considered that it was possible that expressions that contain those emoticons represented only a part of the subjective languages on Twitter.
We asked some people who speaks one of the four languages to help this thesis and one of them who is a native Spanish speaker said the tweets of the modified Spanish training dataset was less correct than the original. 
This happened not only for Spanish but other languages.
Therefore we chose to use only the dataset from \cite{dataset} and \cite{dataset_spanish} without any modification.
%Barcelona 04 Nov 



%\section{Timeline Graph} \label{sec:discussion}

\section{Result of Chi Square Test}

