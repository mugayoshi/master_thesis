\chapter{Introduction}
In this section, we describe motivation, purpose and composition of this thesis.

\section{Motivation} \label{sec:motivation}
As smartphone becomes more popular nowadays, Social Network Service (SNS) has been used by many people around the world.
One of the most popular SNS is Facebook and there are over 1.79 billions monthly active users, which is a 16 \% increase year over year \cite{facebook_user}.
As another example of SNS, Twitter, one of the biggest micro blogging service, has averagely 317 million active user as of the third quarter of 2016 \cite{twitter_user}.
%Even though the numbers of active users are larger than a population of most of countries, that does not means 
Although tweets from one user mostly are not important, thousands of the collection of tweets or opinions can become comprehensive.
Opinion mining helps companies, for example, to get feedback on their products.
Not only feedback, there can be other ways to utilise opinions from SNS.
Sentiment analysis determines if a given sentence is positive, negative or neutral for instance.
Therefore with this system SNS can be a platform where researchers investigate what impressions of certain things (e.g. entertainment events, products, politics) users on SNS have.

%sentiment analysis 

Location information is an important factor in some situations.
One of such situations is national vote.
There was a national vote in the U.K. in June 2016 to decide if the country would leave European Union (EU).
The result turned out that people in London and Scotland supported for remain while those who in the other areas did for leave \cite{uk_referendum}.
A similar phenomena was also observed in the U.S. presidential election in November 2016 \cite{us_map}.
The results of these two votes showed that people even in the same country have different and opposite opinions depending on where they live.
And it was figured out that huge gap of public opinion exits in these two countries.
Lack of understanding people in local areas made the results for the votes unpredictable.
%EU referendum in U.K and presidential election in U.S showed the huge gap exist in these 2 countries.
This issue should be discussed not only the two countries mentioned above because there are many countries where citizens in big cities face completely different circumstances from those who in local cities.
%More people from around the world are likely to come to big cities because of globalization, which contributes to such public opinion gap.
Hence, sentiment analysis by location can be valuable to find out differences of people's opinion. 
%when because it can assist to figure out public sentiment by each area or city.

Twitter is internationally popular in a wide variety of languages.
For example, this service has more active users than Facebook in Japan \cite{japan_twitter}.
A psychology research \cite{psychology1} shows that the language effects on cognition.
Bilingual participants in a German testing context prefer to match events on the basis of motion completion to a greater extent than do bilingual participants in an English context.
%When bilingual participants experience verbal interference in English, their categorization behaviour is congruent with that predicted for German; when bilingual participants experience verbal interference in German, their categorization becomes congruent with that predicted for English. These findings show that language effects on cognition are context-bound and transient, revealing unprecedented levels of malleability in human cognition.
According to another research \cite{psychology2}, participants are more likely to bet on gamble if it is high possibility to get money when they are asked in their foreign language.
This means that the participants think in more rational way when they use their foreign language.
These findings suggest that analysis for tweets in only one language covers only a part of the available content.
%Especially in international big cities, multi-language sentiment analysis can show more than   
%especially in the internet, English is the most common language.

\section{Purpose}
We propose multi-language sentiment analysis across different cities in this thesis.
In order to achieve this, we retrieve tweets by city and classify them by classifiers trained with machine learning. 
We compare the classification results between cities and explore differences between the cities and the languages. 
%And we show the validity of our system using a statistical test, chi square test.

%to explore the distribution of sentiment value of 2 cities
\section{Composition of The Thesis}

\begin{description}
 \item[Chapter 2] explains the prior knowledge to understand this thesis.
 %background.
 \item[Chapter 3] describes related works.
 \item[Chapter 4] details the multi-language sentiment analysis across different cities.
 \item[Chapter 5] explains the experiment.
 \item[Chapter 6] discusses the results of the experiment.
 \item[Chapter 7] concludes this thesis and describe future work.
\end{description}

