\chapter{Introduction}
In this section, we describe the motivation of this thesis, the purpose and the composition of this thesis.

\section{Motivation} \label{sec:motivation}
As smartphone becomes more popular nowadays, Social Network Service (SNS) has been used by many people around the world.
One of the most popular SNS is Facebook and there are over 1.79 billions monthly active users, which is a 16 \% increase year over year \cite{facebook_user}.
As another example of SNS, Twitter, one of the biggest micro blogging service, has averagely 317 million active user as of the third quarter of 2016 \cite{twitter_user}.
%Even though the numbers of active users are larger than a population of most of countries, that does not means 
Although tweets from one user mostly are not important, thousands of the collection of tweets or opinions can become comprehensive.
Opinion mining helps companies, for example, to get feedback on their products.
Not only this, there can be other ways of using opinions from SNS.
Sentiment analysis on SNS determines if a given sentence is positive, negative or neutral for instance.
Therefore with this system SNS can be a platform where researchers investigate what impressions of certain things, which are entertainment events, products, politics etc., the users have.

%sentiment analysis 

Location information is an important factor in many kinds of situations.
One of such situations is national vote.
There was a national vote in U.K in June 2016 to decide if the country would leave EU or not.
People in London and Scotland supported for remain while those who in the other areas did for leave \cite{uk_referendum}.
This phenomena was also observed in U.S presidential election in November 2016 \cite{us_map}.
The results of these 2 votes turned out that people even in the same country have different opinions depending where they live and they showed that huge gap of public opinion exits in these 2 countries.
Not knowing sentiment of people in local areas made the results unpredictable.
%EU referendum in U.K and presidential election in U.S showed the huge gap exist in these 2 countries.
This issue should be discussed not only the 2 countries mentioned above because there are many countries where citizens in big cities face completely different circumstances from those who in local cities.
%More people from around the world are likely to come to big cities because of globalization, which contributes to such public opinion gap.
Sentiment analysis by location can be valuable to find out differences of people's opinion. 
%when because it can assist to figure out public sentiment by each area or city.

Twitter is internationally popular in a wide variety of languages.
As an example, this service has more active users than Facebook in Japan \cite{japan_twitter}.
A psychology research \cite{psychology1} shows that the language effects on cognition.
Bilingual participants in a German testing context prefer to match events on the basis of motion completion to a greater extent than do bilingual participants in an English context.
%When bilingual participants experience verbal interference in English, their categorization behaviour is congruent with that predicted for German; when bilingual participants experience verbal interference in German, their categorization becomes congruent with that predicted for English. These findings show that language effects on cognition are context-bound and transient, revealing unprecedented levels of malleability in human cognition.
According to another research \cite{psychology2}, participants are more likely to bet on gamble if it is high possibility to get money when they are asked in their foreign language.
This means that the participants think in more rational way when they use foreign language.
These findings suggest that analysing tweets in only one language covers only a part of the available content.
%Especially in international big cities, multi-language sentiment analysis can show more than   
%especially in the internet, English is the most common language.

\section{Purpose}
We propose the multi-language sentiment analysis across different cities in this thesis.
In order to achieve this, we retrieve tweets by city and classify them by machine learning. 
We compare the results of the classifications between cities and explore differences or tendency of the cities. 
As evaluation, we show its validity using a statistical test.

%to explore the distribution of sentiment value of 2 cities
\section{Composition of The Thesis}

\begin{description}
 \item[Chapter 2] We explain the prior knowledge to understand this thesis in this chapter.
 %background.
 \item[Chapter 3] We describe the related works in this chapter
 \item[Chapter 4] We detail the multi-language sentiment analysis across different cities in this chapter.
 \item[Chapter 5] We explain the experiment in this chapter.
 \item[Chapter 6] We discuss the results of the experiment in this chapter..
 \item[Chapter 7] We conclude this thesis and mention future work in this chapter.
\end{description}

